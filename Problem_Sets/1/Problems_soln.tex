\documentclass{article}
\usepackage{amsmath,amssymb}
\usepackage{fullpage}
\newcommand{\dsum}{\displaystyle\sum}
\newcommand{\dbcup}{\displaystyle\bigcup}
\newcommand{\dbcap}{\displaystyle\bigcap}
\newcommand{\dcup}{\displaystyle\cup}
\newcommand{\dcap}{\displaystyle\cap}
\newcommand{\Pb}{\mathbb{P}}
\newcommand{\Zb}{\mathbb{Z}}
\newcommand{\bkt}[1]{\left(#1\right)}
\newcommand{\soln}[1]{\textbf{Solution}: #1}
\title{MA2040: Probability, Statistics and Stochastic Processes\\
Problem Set-I}
\author{Sivaram Ambikasaran}
\begin{document}
	\maketitle
	\begin{enumerate}
		\item
		A six-sided die is made in a way that each even face is twice as likely as each odd face. All even faces are equally likely, as are all odd faces. Construct a probabilistic model for a single roll of this die and find the probability that the outcome is less than 4.\\
		\soln{The sample space is $\Omega=\{1,2,3,4,5,6\}$. Let the probability of each odd face be $p$. We then have the probability of each even face to be $2p$. We have
		$$1 = \Pb\bkt{\Omega} = \Pb\bkt{\{1\}} + \Pb\bkt{\{2\}} + \Pb\bkt{\{3\}} + \Pb\bkt{\{4\}} + \Pb\bkt{\{5\}} + \Pb\bkt{\{6\}}$$
		This gives us that
		$$9p=1 \implies p =1/9$$
		$$\Pb\bkt{\text{Outcome is less than }4} = \Pb\bkt{\{1\}} + \Pb\bkt{\{2\}} + \Pb\bkt{\{3\}} = p + 2p + p = 4p = 4/9$$
		}
		\item
		Let $S_1,S_2,\ldots,S_n$ be a partition of the sample space $\Omega$.
		\begin{enumerate}
			\item
			Show that for any event $A$,
			$$\Pb(A) = \dsum_{i=1}^n \Pb\bkt{A \dcap S_i}$$
			\soln{Since $S_1,S_2,\ldots,S_n$ is a partition of $\Omega$, we have $\Omega = \dbcup_{i=1}^n S_i$ and $S_i \dbcap S_j = \emptyset$, whenever $i \neq j$. We have
			$$A = A \dbcap \Omega = A \dbcap \bkt{\dbcup_{i=1}^n S_i} = \dbcup_{i=1}^n \bkt{A \dbcap S_i}$$
			Since $S_i$'s are mutually disjoint, so are $A \dbcap S_i$'s. Hence, we have that
			$$\Pb\bkt{A} = \Pb\bkt{\dbcup_{i=1}^n \bkt{A \dbcap S_i}} = \dsum_{i=1}^n \Pb\bkt{A \dbcap S_i}$$
			}
			\item
			Use the previous part to show that, for events $A$, $B$ and $C$,
			$$\Pb\bkt{A} = \Pb\bkt{A \dcap B} + \Pb\bkt{A \dcap C} + \Pb\bkt{A \dcap B^c \dcap C^c} - \Pb\bkt{A \dcap B \dcap C}$$
			\soln{
			Note that our sample space can be paritioned into two disjoint sets $S_1= B \dbcup C$ and $S_2 = \bkt{B \dbcup C}^c = B^c \dbcap C^c$.
			From previous part, we hence have
			\begin{align}
				\Pb\bkt{A} & = \Pb\bkt{A \dcap S_1} + \Pb\bkt{A \dcap S_2} = \Pb\bkt{A \dcap \bkt{B \dcup C}} + \Pb \bkt{A \dcap \bkt{B^c \dcap C^c}}\\
				& = \Pb\bkt{\bkt{A \dcap B} \dcup \bkt{A \dcap C}} + \Pb \bkt{A \dcap B^c \dcap C^c}\\
				& = \Pb\bkt{A \dcap B} + \Pb\bkt{A \dcap C} - \Pb\bkt{\bkt{A \dcap B} \dcap \bkt{A \dcap C}} + \Pb \bkt{A \dcap B^c \dcap C^c}\\
				& = \Pb\bkt{A \dcap B} + \Pb\bkt{A \dcap C} + \Pb\bkt{A \dcap B^c \dcap C^c} - \Pb\bkt{A \dcap B \dcap C}
			\end{align}
			}
		\end{enumerate}
		\item
		\begin{enumerate}
			\item
			Prove that for any two events $A$ and $B$, we have
			$$\Pb\bkt{A \dcap B} \geq \Pb(A) + \Pb(B) - 1$$
			\soln{
			Recall that $\Pb\bkt{A \dcap B} = \Pb(A) + \Pb(B) - \Pb\bkt{A \dcup B}$. Since $\Pb\bkt{A \dcup B} \leq 1$, we obtain that
			$\Pb\bkt{A \dcap B} \geq \Pb(A) + \Pb(B) - 1$.
			}
			\item
			Using the above, establish the following generalization:
			$$\Pb\bkt{A_1 \dcap A_2 \dcap \cdots \dcap A_n} \geq \Pb\bkt{A_1} + \Pb\bkt{A_2} + \cdots + \Pb\bkt{A_n} - \bkt{n-1}$$
			\soln{
			This is immediately obtained by induction. Let $S_{n+1} = A_1 \dcap A_2 \dcap \cdots A_{n}$. From the previous part, we have
			$$\Pb\bkt{S_{n+1}} = \Pb \bkt{S_n \dcap A_n} \geq \Pb\bkt{S_n} + \Pb\bkt{A_n} - 1$$
			And now by the induction, we have that $\Pb\bkt{S_n} \geq \dsum_{i=1}^{n-1} \Pb \bkt{A_i} - \bkt{n-2}$ and hence we obtain what we want.
			}
		\end{enumerate}
		\item
		Let $\Omega = \{\bkt{x_1,x_2,x_3}: x_1+x_2+x_3 = 2019\}$ and $x_1,x_2,x_3$ are positive integers. Assuming all the elements in $\Omega$ are equally likely to be drawn, what is the probability of having $x_1,x_2$ and $x_3$ (all three) to be odd?\\
		\soln{Let $A$ be the event we are after, i.e., $A = \{\bkt{x_1,x_2,x_3} \in \Omega: x_1,x_2,x_3 \text{ are all odd}\}$. The number of elements in $\Omega$ is given by $\dbinom{2019-1}{3-1} = \dbinom{2018}2$. To obtain the number of elements in $A$, let $x_i = 2y_i-1$, where $y_i \in \Zb^+$. We then need the number of positive integer solutions to
		$$\bkt{2y_1-1} + \bkt{2y_2-1} + \bkt{2y_3-1} = 2019 \implies y_1 + y_2 + y_3 = 1011$$
		The number of elements in $A$ is therefore given by $\dbinom{1011-1}{3-1} = \dbinom{1010}2$. Hence, the desired probability is
		$$\dfrac{\dbinom{1010}2}{\dbinom{2018}2} = \dfrac{1010 \times 1009}{2018 \times 2017} = \dfrac{505}{2017}$$
		\textbf{Remark}: Note that the number of distinct positive integer solutions to $x_1+x_2+\cdots+x_n = N$, where $N \geq n$, $n,N \in \Zb^+$ is given by $$\dbinom{N-1}{n-1}$$
		The proof is as follows. Consider a sequence of $N$ ones, i.e., $\underbrace{111 \ldots 1}_{N \text{ ones}}$. A $n$-partition of $N$ is obtained by inserting $n-1$ bars in the $N-1$ gaps corresponding to $N$ ones. For example, consider $n=3$ and $N=6$. One possible solution is $(x_1,x_2,x_3) = (2,3,1)$. This can be interpreted as inserting a bar after the second $1$ and another bar $3$ ones later, i.e.,
		$$1 1 \Bigg\vert 1 1 1 \Bigg\vert 1$$
		Note that any insertion of $n-1$ bars in the $N-1$ gaps gives a solution (comprising of positive integers) to $x_1+x_2+\cdots+x_n = N$ and any solution (comprising of positive integers) to $x_1+x_2+\cdots+x_n = N$ can be interpreted as inserting $n-1$ bars in the $N-1$ gaps. Hence, the number of positive integer solutions to $x_1+x_2+\cdots+x_n = N$ is given by
		$$\dbinom{N-1}{n-1}$$
		}
		
		\item
		Two fair $6$-sided dice are rolled.
		\begin{enumerate}
			\item
			Given that the roll results in a sum of $4$ or less, find the conditional probability that both dice show the same number.\\
			\soln{$A$ be the event that both dice show the same number and $B$ be the event that the roll results in a sum not exceeding $4$. We need $\Pb \bkt{A \vert B}$. We have $B = \{(1,1),(1,2),(2,1),(2,2)\}$. We have $A = \{(1,1),(2,2),(3,3),(4,4),(5,5),(6,6)\}$ and $A \dcap B = \{(1,1),(2,2)\}$. Hence,
			$$\Pb\bkt{A \vert B} = \dfrac{\Pb \bkt{A \dcap B}}{\Pb\bkt{B}} = \dfrac{2/36}{4/36} = \dfrac12$$}
			\item
			Given that the two dice land on different numbers, find the conditional probability that at least one die roll is a $6$.\\
			\soln{Let $C$ be the event that at least one die roll is $6$ and $D$ be the event that the two dice land on different numbers. We have
			$$C = \{(1,6),(2,6),(3,6),(4,6),(5,6),(6,6),(6,5),(6,4),(6,3),(6,2),(6,1)\}$$
			$$D = \Omega-\{(1,1),(2,2),(3,3),(4,4),(5,5),(6,6)\}$$
			Hence,
			$$\Pb \bkt{C \vert D} = \dfrac{\Pb \bkt{C \dcap D}}{\Pb\bkt{D}} = \dfrac{10/36}{30/36} = 1/3$$
			}
		\end{enumerate}
		\item
		A batch of $100$ items is inspected by testing $4$ randomly selected items. If one of the four is defective, the batch is rejected. What is the probability that the batch is accepted, if it contains exactly five defectives?\\
		\soln{
		\begin{align}
			\Pb\bkt{\text{Batch being accepted}} & = \Pb\bkt{\text{Four randomly selected items being non-defective}}\\
			& = p_1 p_2 p_3 p_4
		\end{align}
		where $p_i = \Pb \bkt{i^{th} \text{ item being non-defective given that the previous }i-1 \text{ items are defective}}$. We have $p_1= \dfrac{95}{100}, p_2 = \dfrac{94}{99}, p_3 = \dfrac{93}{98}$ and $p_4 = \dfrac{92}{97}$. Hence, the desired probability is $\dfrac{95}{100} \times \dfrac{94}{99} \times \dfrac{93}{98} \times \dfrac{92}{97}$.
		}
		\item
		Consider a coin that comes up heads with probability $p$ and tails with probability $1-p$. Let $q_n$ be the probability of obtaining even number of heads in $n$ independent tosses. Derive a recursion that relates $q_n$ to $q_{n-1}$ and establish the formula
		$$q_n = \dfrac{1+\bkt{1-2p}^n}2$$
		\soln{
		We will condition on the $n^{th}$ toss.
		\begin{align}
			q_n & = \Pb \bkt{\text{Obtaining even number of heads in $n$ independent tosses }\vert\text{ Last toss is a head}} \Pb \bkt{\text{Last toss is a head}}\\
		& + \Pb \bkt{\text{Obtaining even number of heads in $n$ independent tosses }\vert\text{ Last toss is a tail}} \Pb \bkt{\text{Last toss is a tail}}\\
		& = \Pb\bkt{\text{Obtaining odd number of heads in $n-1$ independent tosses}} \cdot \Pb \bkt{\text{Last toss is a head}}\\
		& + \Pb\bkt{\text{Obtaining even number of heads in $n-1$ independent tosses}} \cdot \Pb \bkt{\text{Last toss is a tail}}\\
		& = \bkt{1-q_{n-1}}p + q_{n-1} \bkt{1-p} = p + \bkt{1-2p}q_{n-1}
		\end{align}
		Note that $q_1=1-p$. Hence, we have
		\begin{align}
			q_n & = p + (1-2p)\bkt{p+\bkt{1-2p}q_{n-2}} = p + p \bkt{1-2p} + \bkt{1-2p}^2 q_{n-2}\\
			& = p + p\bkt{1-2p} + p \bkt{1-2p}^2 + p \bkt{1-2p}^3 + \cdots + \bkt{1-2p}^{n-1} q_1\\
			& = p \dsum_{k=0}^{n-2} \bkt{1-2p}^k + (1-p)\bkt{1-2p}^{n-1}\\
			& = p \bkt{\dfrac{1-\bkt{1-2p}^{n-1}}{2p}} + \bkt{1-p}\bkt{1-2p}^{n-1}\\
			& = \dfrac{1-\bkt{1-2p}^{n-1}}2 + \dfrac{2\bkt{1-p}\bkt{1-2p}^{n-1}}2\\
			& = \dfrac{1 + \bkt{1-2p}^{n-1}\bkt{2-2p-1}}2\\
			& = \dfrac{1+\bkt{1-2p}^n}2
		\end{align}
		}
		\item
		Two playes $X$ and $Y$ alternately roll a pair of unbiased dice. $X$ wins if on a throw he gets a sum of $6$ before $Y$ gets a sum of $7$; $Y$ wins if he obtains a sum of $7$ before $X$ obtains a sum of $6$; If $X$ begins the game, prove that his probability of winning is $30/61$.\\
		\soln{
		$$\Pb\bkt{X \text{ winning}} = \dsum_{n=1}^{\infty}\Pb\bkt{X \text{ winning on the $n^{th}$ attempt}}  = \dsum_{n=1}^{\infty}p_n$$
		$$p_n = \Pb\bkt{\text{$X$ doesn't roll a sum of $6$ and Y doesn't roll a sum of $7$ in the previous $n-1$ attempts}}$$
		$$p_n = \dsum_{n=1}^{\infty} p^{n-1}q^{n-1} \bkt{1-p}$$
		where $p$ is the probability of not getting a sum of $6$ and $q$ is the probability of not getting a sum of $7$. We have $p=1-5/36 = 31/36$ and $q=1-6/36 = 5/6$. Hence, the desired probability is
		$$\dsum_{n=1}^{\infty} \bkt{\dfrac{31}{36}}^{n-1} \bkt{\dfrac{5}6}^{n-1} \times \dfrac5{36} = \dfrac5{36} \times \dfrac1{1-\dfrac{155}{216}} = \dfrac{30}{61}$$
		}
		\item
		In a deck of cards, let $A$ be the event of drawing a spade and $B$ be the event of drawing a king. Assuming that all cards are equally likely to be drawn, obtain $\Pb(A \vert B)$, $\Pb (B \vert A)$. Are $A$ and $B$ independent?\\
		\soln{
		$\Pb(A) = \dfrac{13}{52} = \dfrac14$, $\Pb(B) = \dfrac4{52} = \dfrac1{13}$, $\Pb\bkt{A \dcap B} = \dfrac1{52}$, $\Pb\bkt{A \vert B} = \dfrac{\Pb\bkt{A \dcap B}}{\Pb\bkt{B}} = \dfrac{1/52}{4/52} = \dfrac14$, $\Pb \bkt{B \vert A} = \dfrac{\Pb\bkt{A \dcap B}}{\Pb(A)} = \dfrac{1/52}{13/52} = \dfrac1{13}$. Hence, we see that
		$$\Pb\bkt{A \vert B} = \Pb\bkt{A}$$
		and hence $A$, $B$ are independent.
		}
		\item
		Let $p_X$ denote the probability that India will play against team $X$ in the World Cup Final $2019$, given that India will qualify for the World Cup Final $2019$. Let $q_X$ be the probability of Kohli scoring a century against team $X$.

		\begin{table}[!htbp]
			\begin{center}
			\begin{tabular}{|c|c|c|}
				\hline
				Teams & $p_X$ & $q_X$\\
				\hline
				\hline
				Afghanistan & $0.02$ & $0.50$\\
				\hline
				Australia & $0.10$ & $0.30$ \\
				\hline
				Bangladesh & $0.03$ & $0.40$\\
				\hline
				England & $0.30$ & $0.10$\\
				\hline
				New Zealand & $0.15$ & $0.25$\\
				\hline
				Pakistan & $0.10$ & $0.15$\\
				\hline
				South Africa & $0.20$ & $0.20$\\
				\hline
				SriLanka & $0.05$ & $0.25$\\
				\hline
				West Indies & $0.05$ & $0.20$\\
				\hline
			\end{tabular}
			\end{center}
		\end{table}
		\begin{enumerate}
			\item
			From the information given above, find the probability that Kohli will score a century in the World Cup Final $2019$.\\
			\soln{
			Let $A$ be the event of Kohli scoring a century in the World Cup Final $2019$ and $E_X$ be the event of India playing against $X$ in the World Cup Final $2019$. Note that we have $p_X = \Pb \bkt{E_X}$ and $q_X = \Pb\bkt{A \vert E_X}$.
			\begin{align*}
				\Pb\bkt{A} & = \dsum_{X \in \{\text{all teams}\}}\Pb\bkt{A \vert E_X} \Pb \bkt{E_X} = \dsum_{X \in \{\text{all teams}\}}p_X q_X\\
				& = 0.02 \times 0.50 + 0.10 \times 0.30 + 0.03 \times 0.40 + 0.30 \times 0.10 + 0.15 \times 0.25\\
				& + 0.10 \times 0.15 + 0.20 \times 0.20 + 0.05 \times 0.25 + 0.05 \times 0.20\\
				& = 0.197
			\end{align*}
			}
			\item
			Given that Kohli scores a century in the World Cup Final $2019$, which team is
			\begin{enumerate} 
				\item Most likely to have played the Final along with India
				\item Least likely to have played the Final along with India
			\end{enumerate}
			\soln{We need $\Pb\bkt{E_X \vert A}$. We have
			$$\Pb\bkt{E_X \vert A} = \dfrac{\Pb\bkt{A \vert E_X} \Pb\bkt{E_X}}{\Pb\bkt{A}}$$
			\begin{table}[!htbp]
				\begin{center}
			{\renewcommand{\arraystretch}{2.5}
			\begin{tabular}{|c|c|}
				\hline
				Afghanistan & $\dfrac{0.02 \times 0.50}{0.197} = \dfrac{10}{197}$\\
				\hline
				Australia & $\dfrac{0.10 \times 0.30}{0.197} = \dfrac{30}{197}$\\
				\hline
				Bangladesh & $\dfrac{0.03 \times 0.40}{0.197} = \dfrac{12}{197}$\\
				\hline
				England & $\dfrac{0.30 \times 0.10}{0.197} = \dfrac{30}{197}$\\
				\hline
				New Zealand & $\dfrac{0.15 \times 0.25}{0.197} = \dfrac{37.5}{197}$\\
				\hline
				Pakistan & $\dfrac{0.10 \times 0.15}{0.197} = \dfrac{15}{197}$\\
				\hline
				South Africa & $\dfrac{0.20 \times 0.20}{0.197} = \dfrac{40}{197}$\\
				\hline
				SriLanka & $\dfrac{0.05 \times 0.25}{0.197} = \dfrac{12.5}{197}$\\
				\hline
				West Indies & $\dfrac{0.05 \times 0.20}{0.197} = \dfrac{10}{197}$\\
				\hline
			\end{tabular}
			}
			\end{center}
			\end{table}
			We see that South Africa is the most likely team to have played the Final against India, while West-Indies and Afghanistan are the least likely teams to have played the Final against India.
			}
		\end{enumerate}
		\newpage
		\item
		A test for certain rare virus correctly predicts that the person has a virus $99\%$ of the time and correctly identifies that the person doesn't carry a virus $98\%$ of the time. It is known $1\%$ of the population carries the virus. What is the probability of a person actually having the disease, if he has tested positive to the test?\\
		\soln{
		Let $A$ be the event that test predicts virus present, and $B$ be the event that the person carries the virus. We are given that
		$$\Pb\bkt{A \vert B} = 0.99, \Pb \bkt{A^c \vert B^c} = 0.98, \Pb\bkt{B} = 0.01$$
		We need $\Pb \bkt{B \vert A}$. We have
		$$\Pb\bkt{A \vert B^c} = 1-\Pb\bkt{A^c \vert B^c} = 0.02$$
		We also have that
		$$\Pb \bkt{B^c} = 1- \Pb \bkt{B} = 0.99$$
		$$\Pb\bkt{A} = \Pb\bkt{A \vert B} \Pb \bkt{B} + \Pb\bkt{A \vert B^c} \Pb \bkt{B^c} = 0.99 \times 0.01 + 0.02 \times 0.99 = 0.03 \times 0.99$$
		Hence,
		$$\Pb \bkt{B \vert A} = \dfrac{\Pb\bkt{A \vert B} \Pb\bkt{B}}{\Pb\bkt{A}} = \dfrac{0.99 \times 0.01}{0.03 \times 0.99} = \dfrac13$$
		}
	\end{enumerate}
\end{document}